% Please add the following required packages to your document preamble:
% \usepackage[normalem]{ulem}
% \useunder{\uline}{\ul}{}

%\begin{tabular}{p{0.25\linewidth}p{0.25\linewidth}p{0.25\linewidth}}
\begin{table*}[ht]
\scriptsize
    \centering
    
    \caption{Qualitative Validation of the OCL Highlight Plugin}
    
    \label{tbl:pluginValidation}
    
    \begin{tabular}{cll}
        
        \hline
        
        \\
        
        \multicolumn{1}{l}{}QUESTIONS & 
        %\begin{tabular}[c]{@{}l@{}}
        \begin{tabular}[c]{m{5.5cm}}
            What is your opinion, as a UML/OCL expert, on the usefulness of this plugin? Considering that you have used the tool without the plugin, do you think it can  help students when learning OCL? \\ \\
        \end{tabular}
        
        \begin{tabular}[c]{m{5.5cm}} 
            How did the students in your class(es) react to the plugin? In particular, the assimilation of the visual metaphor was easy, that is the correspondence between the textual expression in OCL and the highlighted elements in the class diagram? \\ \\
        \end{tabular}
        
        \\
        
        \hline
        
        \\
        
        \begin{tabular}[c]{m{1cm}}
            Expert 1
        \end{tabular} &
        
        \begin{tabular}[c]{p{5.5cm}}
            An indispensable and missing plugin, so far! While learning OCL, it is critical to understand the relationship between the often complex OCL expressions and related modeling elements defined in a class diagram – often complex as well. OCL Highlight plugin delivers a simple, yet complete, graphical representation of those relationships, thus facilitating the understanding and learning of OCL expressions.
        \end{tabular}
        
        \begin{tabular}[c]{p{5.5cm}}
            The students quickly understood the usefulness and ease of use of the plugin. The visual matching between the OCL expression and the related modeling elements in the class diagram was considered very intuitive. The graphical feedback also motivated the students to try out and understand increasingly complex OCL expressions.
        \end{tabular} 
        
        \\
        
        \begin{tabular}[c]{m{1cm}}
            Expert 2
        \end{tabular} & 
        
        %\begin{tabular}[c]{@{}l@{}}
        \begin{tabular}[c]{m{5.5cm}}
            I consider that the plugin is of great utility since it illustrates the navigation of the queries in OCL. It facilitates the understanding of the expressions, as well as the understanding of OCL language.
        \end{tabular}
        
        \begin{tabular}[c]{m{5.5cm}}
            The reaction of the students to the plugin was something natural as if it was something obvious that the tool "painted the way". I took the opportunity of asking them if the path was not painted in the class diagram would be more difficult to understand the semantics of queries, and most students said yes.
        \end{tabular}
        
        \\
        
        \begin{tabular}[c]{m{1cm}}
            Expert 3
        \end{tabular} &
        
        \begin{tabular}[c]{m{5.5cm}}
            The plugin is extremely important as students have some difficulties to understand OCL and, more importantly, to build OCL expressions. This visual insight helps to grasp how OCL operates over the model.
        \end{tabular}
        
        \begin{tabular}[c]{m{5.5cm}}
            Yes, they have immediately realized which parts of the model were involved.
        \end{tabular}                                                               
        
        \\
        
                \begin{tabular}[c]{m{1cm}}
            Expert 4
        \end{tabular} &
        
        \begin{tabular}[c]{m{5.5cm}}
            It is a very useful plugin since it helps the students understand what classes are being used in each OCL expression. It introduces traceability and shows a better insight to the students, making it easier for them to learn.
        \end{tabular}
        
        \begin{tabular}[c]{m{5.5cm}}
            Yes, they understood quickly how OCL expressions worked.
        \end{tabular}
        
        \\
    
        \begin{tabular}[c]{m{1cm}}
            Expert 5
        \end{tabular} &
        
        \begin{tabular}[c]{m{5.5cm}}
            This plugin is mostly a good facilitator and a really useful tool for the understanding of UML/OCL, mainly when we have class diagrams with a considerable size. Especially for students, it can ease and improve the process of learning OCL queries.
        \end{tabular}
        
        \begin{tabular}[c]{m{5.5cm}}
            In class and with this plugin, students were able to follow the OCL expressions they were writing by immediately visualizing the result of those queries in terms of the used classes, associations and attributes that were highlighted. Students accepted the use of this tool really well, compared with the previous year, it not only provided a more engaging learning experience for the students but also, as a lecturer, it helped to demonstrate the process of querying.
        \end{tabular} 
        
        \\
        
        \begin{tabular}[c]{m{1cm}}
            Expert 6
        \end{tabular} &
        
        \begin{tabular}[c]{m{5.5cm}}
            I believe that, in fact, this plugin can facilitate the learning of OCL by the students. The selective visualization allows a greater concentration and effectiveness of the analysis of the pathway performed by the queries.
        \end{tabular}
        
        \begin{tabular}[c]{m{5.5cm}}
            The students in my class seemed to assimilate well the connection between the path highlighted in the diagram and the path made by the OCL queries.
        \end{tabular}
        
           \\ \\
        
        \hline
    
    \end{tabular}

\end{table*}

%\addtotoc{Resumo}

Modelos, metamodelos e transformações de modelo desempenham um papel central no Desenvolvimento Dirigido por Modelos (MDD). A Linguagem para Especificação de Restrições em Objetos (OCL) foi inicialmente proposta como parte da Linguagem de Modelagem Unificada (UML) para adicionar os recursos de precisão e validação que faltavam nestes diagramas, e também para expressar regras de boa formação no metamodelo. A OCL possui outras aplicações, tais como definir métricas de desenho, modelos de geração de código ou regras de validação para transformações de modelo, exigidas em MDD. 

Aprender OCL como parte de um curso de UML na universidade parecia portanto natural. No entanto não é o que se verifica. Acreditamos que isso se deva principalmente a uma percepção generalizada de que OCL é difícil de aprender, tendo em conta afirmações feitas na literatura. Com base em dados recolhidos em anos letivos anteriores de um grande número de alunos de licenciatura de diferentes cursos de Engenharia de Software, analisámos como a aprendizagem por cláusulas contratuais de UML + OCL se compara a vários outros tópicos do SWEBOK. O resultado do processo de aprendizagem foi recolhido de forma rigorosa, apoiada por uma plataforma de e-learning. Realizámos estatísticas inferenciais sobre os dados para apoiar as nossas conclusões, de forma a identificar as variáveis explicativas relevantes para o sucesso / fracasso dos alunos. As conclusões obtidas levaram-nos a estender uma ferramenta OCL existente com dois recursos novos: o primeiro é voltado para os estudantes de OCL e vai direto ao núcleo da questão, permitindo visualizar como as expressões OCL percorrem um diagrama de classes UML; o segundo é voltado para investigadores e permite calcular métricas de complexidade OCL, habilitando a réplica de um estudo como este.

% Palavras-chave do resumo em Português
\begin{keywords}
OCL; UML; Realce do modelo; Compreensão do OCL.
\end{keywords}
% to add an extra black line

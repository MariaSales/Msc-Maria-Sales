
\usepackage{url}
\usepackage[]{algorithm2e}
\usepackage{rotating}
\usepackage{multirow}
\usepackage{listings}
\usepackage{amsmath}
\usepackage{amsfonts}
\usepackage{amssymb}
\usepackage{pgfplots}
\usepackage{graphicx}
\usepgfplotslibrary{statistics}
\usetikzlibrary{patterns}
\usepackage{comment}
\usepackage{pgfplotstable}
\usepackage{float}


%--------------------------------------------------
% Adicionado para ter cor nas tabelas
\usepackage{colortbl}
%
%--------------------------------------------------
%
% Adiciona o estilo do capítulo
\usepackage{titlesec, color}
\definecolor{gray75}{gray}{0.75}
\newcommand{\hsp}{\hspace{20pt}}
%\titleformat{\chapter}[hang]{\Huge\bfseries\scshape}{\thechapter\hsp\textcolor{gray75}{|}\hsp}{0pt}{\Huge\bfseries}
%
%--------------------------------------------------
% Formatação das páginas 
\pagestyle{fancy}
\fancyhf{}
\fancyhead[LE,RO]{\slshape \leftmark}

%\fancyfoot[LE,RO]{\thepage}


\fancypagestyle{plain}{% % <-- this is new
  \fancyhf{} 
  \fancyfoot[LE,RO]{\thepage} % same placement as with page style "fancy"
  \renewcommand{\headrulewidth}{0pt}}
  \renewcommand{\chaptermark}[1]{\markboth{\thechapter.\ #1}{}}
%
%--------------------------------------------------
%  
% Package para ter glossário
\usepackage[nonumberlist]{glossaries} % glossário sem numeração das páginas em que aparecem as siglas
\makeglossaries
%\robustify{\gls}
%
%--------------------------------------------------
% Package para ter índice de equações
\usepackage{tocloft}
\makeatother
% Criar a listas de equações
\numberwithin{equation}{chapter}
\setcounter{secnumdepth}{3}
%
%--------------------------------------------------

\newcommand*\rfrac[2]{{}^{#1}\!/_{#2}}
\newcommand{\tuple}[1]{\langle #1\rangle}
\newcommand{\fromRoot}[1]{./data/#1}

\def \listfigures {List of Figures}

\def \university {University Institute of Lisbon}

\def \department {Department of Information Science and Technology}

\def \supervisortitle {Supervisor}
\def \supervisorname {Your Supervisor's Name, Assistant or Associate or Full Professor}
\def \supervisoruniv{ISCTE-IUL}

%Se tiverem co-orientador basta descomentar esta parte seguinte

%\def \cosupervisortitle {Co-supervisor}
%\def \cosupervisorname {Your Co-supervisor's Name, Assistant or Associate or Full Professor}}
%\def \cosupervisoruniv {FCT/UNL}

\def \dissertationstatement {Dissertation in partial fulfillment of the requirements for the degree of}
\def \dissertationtitle {M.Sc. in Computer Science and Business Management}
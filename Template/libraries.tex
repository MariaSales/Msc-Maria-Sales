%%%%%%%%%%%%%%%%%%%%%%%%%%%%%%%%%%%%%%%%%%%%%%%%%%%%%%%%%%%%%%%%%%%%%%%%
% Part - When document is split into parts
%%%%%%%%%%%%%%%%%%%%%%%%%%%%%%%%%%%%%%%%%%%%%%%%%%%%%%%%%%%%%%%%%%%%%%%%
%% This is the code to include document PARTs
%% and in the back of the PART page includes an graphic about the document 
%% structure
\makeatletter
\renewcommand\part{%
   \if@openright
     \cleardoublepage
   \else
     \clearpage
   \fi
   \thispagestyle{empty}%
   \if@twocolumn
     \onecolumn
     \@tempswatrue
   \else
     \@tempswafalse
   \fi
   \null\vfil
   \secdef\@part\@spart}
\makeatother

\newcommand{\partcover}[3]{ 
\nopartblankpage
\part{#1}

\thispagestyle{plain}
\begin{center}
   \includegraphics[width=14cm]{Chapters/figures/#2}
\end{center}
%\cleardoublepage % or \clearpage Text on the following blank page \clearpage
%\clearpage
\vfill
\par\noindent\rule{\textwidth}{0.4pt}
#3
\par\noindent\rule{\textwidth}{0.4pt}
}
%%%%%%%%%%%%%%%%%%%%%%%%%%%%%%%%%%%%%%%%%%%%%%%%%%%%%%%%%%%%%%%%%%%%%%%%
% Custom/New chapter start
%%%%%%%%%%%%%%%%%%%%%%%%%%%%%%%%%%%%%%%%%%%%%%%%%%%%%%%%%%%%%%%%%%%%%%%%
\newcommand{\chaptercover}[3]{ 
\chapter{#1}
\label{#2}
\minitoc

\vfill
\par\noindent\rule{\textwidth}{0.4pt}
#3
\par\noindent\rule{\textwidth}{0.4pt}
\newpage
}

%%%%%%%%%%%%%%%%%%%%%%%%%%%%%%%%%%%%%%%%%%%%%%%%%%%%%%%%%%%%%%%%%%%%%%%%
% Quote at the beginning of each chapter
%%%%%%%%%%%%%%%%%%%%%%%%%%%%%%%%%%%%%%%%%%%%%%%%%%%%%%%%%%%%%%%%%%%%%%%%
\newcommand{\chapterquote}[3]{ 
\begin{flushright}
\rightskip=0.8cm\textit{``#1''} \\
\vspace{.2em}
\rightskip=.8cm---#2\footnote{#3}
\end{flushright}
\vspace{1em}
}

\newcommand{\chapterquotes}[4]{ 
\begin{flushright}
\rightskip=0.8cm\textit{``#1''\footnote{#2}}\\
\vspace{.2em}
\rightskip=.8cm---#3\footnote{#4}
\end{flushright}
\vspace{1em}
}


\newcommand{\musicquote}[2]{ 
%\footnotesize \\Companion Soundtrack: \href[pdfnewwindow=true]{#1}{#2}
%\normalsize
}


\newcommand{\tikzcircle}[2][red,fill=red]{\tikz[baseline=-0.5ex]\draw[#1,radius=#2] (0,0) circle;}%

\newcommand{\piezero}[1]{%
  \begin{tikzpicture}
    \draw (0,0) circle (1ex); \fill[white] (1ex,0) arc (0:#1:1ex) -- (0,0) -- cycle;
  \end{tikzpicture}%
}

\newcommand{\pie}[1]{%
  \begin{tikzpicture}
    \draw (0,0) circle (1ex); \fill[black!90] (1ex,0) arc (0:#1:1ex) -- (0,0) -- cycle;
  \end{tikzpicture}%
}

\newcommand{\pietwo}[1]{%
\begin{tikzpicture}[rotate=270]
 \draw (0,0) circle (1ex); \fill[black!90] (1ex,0) arc (0:#1:1ex) -- (0,0) -- cycle;
\end{tikzpicture}%
}

\newcommand{\piethree}[1]{%
\begin{tikzpicture}[rotate=180]
 \draw (0,0) circle (1ex); \fill[black!90] (1ex,0) arc (0:#1:1ex) -- (0,0) -- cycle;
\end{tikzpicture}%
}


\newcommand{\zeropct}{\piezero{0}}

\newcommand{\quarterpct}{\pie{90}}

\newcommand{\halfpct}{\pietwo{180}}

\newcommand{\threequarterpct}{\piethree{270}}

\newcommand{\fullpct}{\pie{360}}


\newcommand{\classboard}[6]{

\begin{table}[h]
	\caption{#1}
	%\label{#2}
\centering
\begin{tabular}{lp{0.3cm}p{8cm}}
    %\caption{#1}
	%\label{#2}
	\hline\noalign{\smallskip}
     &  & \textbf{Description}
	\\\noalign{\smallskip}\hline\noalign{\smallskip}
	 \textbf{The benefit is:} &  & \\[0.2cm]
	 \textbf{Absent (0)} & \zeropct & #2\\[0.2cm]
     \textbf{Weak (0.25)} & \quarterpct & #3\\[0.2cm]
     \textbf{Moderate (0.5)} & \halfpct & #4\\[0.2cm]
     \textbf{Strong (0.75)} & \threequarterpct & #5\\[0.2cm]
     \textbf{Complete (1)} & \fullpct & #6\\
	 \noalign{\smallskip}\hline
\end{tabular}
\end{table}

}

\newcommand{\noteboard}[2]{
\vspace{0.5cm}
\begin{tcolorbox}[width=\textwidth,colback={white},title={\textbf{#1}},colbacktitle=gray!10,coltitle=black, boxrule=0.5pt, leftrule=3pt, sharp corners=all]  % auto outer -> makes rounded corners
    #2
\end{tcolorbox}
\vspace{1cm}
}


\definecolor{iscte-iul-palette}{HTML}{14BFB8} % ISCTE - ISTA Color, change it to your desired color. DON'T CHANGE THE VARIABLE NAME.



\newcommand{\hypnull}{
\textbf{H\textsubscript{0}}\xspace
}

\newcommand{\hypalt}{
\textbf{H\textsubscript{1}}\xspace
}



